\documentclass{article}
\usepackage{amsmath, amsthm, amssymb}

\newtheorem{problem}{Problem}
\theoremstyle{definition}
\newtheorem*{solution}{Solution}

\begin{document}
\title{Assignment 3}
\author{Siddharth Bhat}
\date{\today}
\maketitle

\begin{problem}
    Prove or disprove: $f(n) = O(g(n)) \implies 2^{f(n)} = O(2^{g(n)})$
\end{problem}
\begin{solution}
    Assume the given statement is true. 
    Consider $f(n) = 2n$, $g(n) = n$. Clearly, $f(n) \in O(g(n))$.
    Now,
    \begin{gather*}
        2^{f(n)} = 2^{2n}= (2^2)^n = 4^n \\
        O(2^{g(n)}) = O(2^n) \\
        \text{Since from our hypothesis} \\
        2^{f(n)} \in O(2^{g(n)}) \\
        4^n \in O(2^n) \\
        \text{This implies that} \exists N \in \mathbb N,\ \exists c \in \mathbb R \text{ where $c$ is a constant, such that }
        \forall i \geq N, \ 4^n = c \cdot 2^n \\
        \left ( \frac{4}{2} \right ) ^n = c \implies \\
        c = 2^n \\
        \text{Hence, $c$ is not a constant, violating our assumption.}
    \end{gather*}

    Therefore, the theorem is incorrect.
\end{solution}

\begin{problem}
    Prove or disprove: $f(n) + g(n) = \Theta(min \{ f(n), g(n) \})$
\end{problem}
\begin{solution}
    Given statement is false.

    let $f(n) = n$, $g(n) = 2^n$. Consider the problem statement to be
    true. This gives us

    \begin{gather*}
    f(n) + g(n) = n + 2^n \\
    \text{from the theorem, } \\
    f(n) + g(n) \in \Theta(min \{ f(n), g(n) \}) = \Theta(min \{n, 2^n \}) = \Theta(n) \\
    \text{which implies that} \\
    n + 2^n \in O(n) \\
    \text{which is false, since $2^n$ grows rapidly faster than $n$, which can
    be checked with a little algebra.}
    \end{gather*}
    Hence, by contradiction, the statement $f(n) + g(n) = \Theta(min \{ f(n), g(n) \})$ is false.
\end{solution}


\begin{problem}
    Prove or disprove: $f(n) \neq O(g(n)) \implies g(n) = O(f(n))$
\end{problem}
\begin{solution}
    Given statement is false.

    let $f(n) = \sin(n)$, $g(n) = \cos(n)$.


    $f(n) \neq O(g(n))$ since they are periodic and intersect each other
    infinitely many times over $\mathbb R$. Hence, they do not dominate
    one another. Similary, $g(n) \neq O(f(n))$ by the exact same reasoning.
    

    \begin{gather*}
        \text{Algebraically, for a proof by contradiction, assume } g(n) = O(f(n)) \\
        cos(n) = O(sin(n)) \\
        \text{Therefore, by definition of $O(\cdots)$ notation, } \\
        \exists N \in \mathbb N,
        \exists c \in \mathbb R
        \text{ where $N$ and $c$ are constants such that } 
        \forall i \geq N, \  \cos(i) \leq c \sin(i) \\
        \text{this implies} \\
        \tan(i) \leq c \\
        \text{ However, that means that c is not a constant with respect to $i$, which is illegal.} \\
        \text{Therefore, } g(n) \neq O(f(n))
    \end{gather*}

\end{solution}


\begin{problem}
    Prove or disprove: $min \{ f(n), g(n) \} \in \Theta(f(n) + g(n))
\end{problem}
\begin{solution}
\end{solution}

\end{document}
