\documentclass[11pt]{article}
\usepackage{amsmath, amsthm, amssymb}
\usepackage{mathtools}

\newtheorem*{problem}{Problem}
\theoremstyle{definition}
\newtheorem*{solution}{Solution}
\newtheorem{lemma}{Lemma}

\setcounter{secnumdepth}{5} 

\DeclarePairedDelimiter\floor{\lfloor}{\rfloor}


\begin{document}
\title{Algorithms Assignment 1}
\author{Siddharth Bhat (20161105)}
\date{\today}
\maketitle

\section{Prove or disprove the given statements}
\begin{problem}
    Prove or disprove: $f(n) = O(g(n)) \implies 2^{f(n)} = O(2^{g(n)})$
\end{problem}
\begin{solution}
    Assume the given statement is true. 
    Consider $f(n) = 2n$, $g(n) = n$. Clearly, $f(n) \in O(g(n))$.
    Now,
    \begin{align*}
        2^{f(n)} = 2^{2n} &= 4^n \\
        O(2^{g(n)}) &= O(2^n)
    \end{align*}
     Since our hypothesis is that $2^{f(n)} \in O(2^{g(n)})$
    $$4^n \in O(2^n)$$
    
This implies that  $\exists N \in \mathbb N,\ \exists c \in \mathbb R$ where $c$ is a constant, such that
	\begin{align*}
        \forall i \geq N, \ 4^n &= c \cdot 2^n \\
        \left ( \frac{4}{2} \right ) ^n &= c \implies \\
        c &= 2^n \\
    \end{align*}

  Hence, $c$ is not a constant, violating our assumption.  
    Therefore, the theorem is incorrect.
\end{solution}

\begin{problem}
    Prove or disprove: $f(n) + g(n) = \Theta(min \{ f(n), g(n) \})$
\end{problem}
\begin{solution}
    Given statement is false.

    let $f(n) = n$, $g(n) = 2^n$. Consider the problem statement to be
    true. This gives us

    \begin{gather*}
        f(n) + g(n) = n + 2^n \\
        \text{from the theorem, } \\
        f(n) + g(n) \in \Theta(min \{ f(n), g(n) \}) = \Theta(min \{n, 2^n \}) = \Theta(n) \\
        \text{which implies that} \\
        n + 2^n \in O(n) \\
        \text{which is false, since $2^n$ grows rapidly faster than $n$, which can
        be checked with a little algebra.}
    \end{gather*}
    Hence, by contradiction, the statement $f(n) + g(n) = \Theta(min \{ f(n), g(n) \})$ is false.
\end{solution}


\begin{problem}
    Prove or disprove: $f(n) \neq O(g(n)) \implies g(n) = O(f(n))$
\end{problem}
\begin{solution}
    Given statement is false.

    Assume the statement is true. let $f(n) = \sin(n)$, $g(n) = \cos(n)$.

    $f(n) \neq O(g(n))$ since they are periodic and intersect each other
    infinitely many times over $\mathbb R$. Hence, they do not dominate
    one another. Similary, $g(n) \neq O(f(n))$ by the exact same reasoning.
\end{solution}


\begin{problem}
    Prove or disprove: $min \{ f(n), g(n) \} \in \Theta(f(n) + g(n))$
\end{problem}
\begin{solution}
    Given statement is false.

    Assume the statement is true. Let $f(n) = n$, and $g(n) = n^2$. Consider the expression

    \begin{gather*}
        \forall n \in \mathbb N \\
        min \{f(n), g(n) \} = min \{n, n^{2} \} = n \\
        \text{and also} \\
        \Theta(f(n) + g(n)) = \Theta(n + n^{2}) = \Theta(n^{2})
    \end{gather*}
    But we know that
    $$
    n \notin \Theta(n^{2})
    $$

    This violates our assumption that 
    $$
    min \{ f(n), g(n) \} \in (\Theta(f(n) + g(n))
    $$
    Hence, the statement is false
\end{solution}

\section{Solve the following recurrence relations}
\begin{problem}
    Solve the following recurrence relation: $T(n) = 3T\left( \frac n 4 \right) + T \left(\frac n 8 \right) + n \log n$
\end{problem}
\begin{solution}
    We go by the intuition that since the subproblems are of sizes
    $\frac n 4$ and $\frac n 8$, the sum of sizes of subproblems does not
    exceed $n$. Hence, the problem size at each level ($n \log n$) will
    dominate.

    Assuming $T(n) \in O(n \log n) gives$

    \begin{align*}
        T(n) &= 3T \left( \frac n 4 \right) + T \left(\frac n 8 \right) + n \log n \\
        T(n) &= 3O \left( \frac n 4 \log{\frac n 4} \right)  O \left(\frac n 8 \log{\frac n 8} \right) + n \log n \\
        T(n) &= O \left( n \log n \right) \tag{$ O \left( n \log n \right)$ dominates the other two complexities}
    \end{align*}

    Hence, $T(n) \in O(n \log n)$ and our assumption is correct

\end{solution}

\begin{problem}
    Solve the following recurrence relation: $T(n) = T(n - 10) + log(n)$
\end{problem}
\begin{solution}
    I make the assumption that
    \begin{gather*}
        \forall i, 0 \leq i \leq 9, T(i) = 1
    \end{gather*}

    Let $n = 10k + r, k, r \in \mathbb N, \ 0 \leq r < 10$ (by division algorithm). Now, reducing the recurrence relation,
    \begin{align*}
        T(n) &= T(n - 10) + \log n \\ß
        T(n) &= (T(n - 20) + \log n) + \log n \\
        	 &= T(n - 2 \cdot 10) + 2 \log n \\
        \text{generalizing to } p &\in \mathbb N \text{ gives us} \\
        T(n) &= T(n - 10p) + p \log n  \\
        T(10k + r) &= T(10k + r - 10p) + p \log n \\
        \text{let } p &= k \text{ to solve the recurrence} \\
        T(10k + r) &= T(10k + r -10k) + k \log n \\
        T(10k + r) &= T(r) + k \log n \\
        T(10k + r) &= 1 + k \log n \\
        \text{but } k &= \frac{(n - r)}{10} = \\
        			 &\floor*{\frac{n}{10}} \text{. Hence} \\
        T(n) &= 1 + \floor*{\frac{n}{10}} \log n
    \end{align*}
\end{solution}


\begin{problem}
    Solve the following recurrence: $T(n) = 4T(\frac n 3) + n^{\frac 5 4}$
\end{problem}
\begin{solution}
    Applying master's theorem on the given problem, with respect to the standard form
    $$
    T(n) = aT \left( \frac{n}{b} \right) + f(n) \\
    $$
    where $a = 4$, $b = 3$, $f(n) = n^{\frac 5 4} = n^{1.25}$.
    To apply Masters's Theorem, we need to consider the comparison value
    $$n^{\log_b a} = n^{log_3 4} = n^{1.26...}$$
    Since
    $$
    f(n) = n^{1.25} \in O(n^{\log_b a - \epsilon} = n^{1.26 - \epsilon})
    $$
    We can show the existence of $\epsilon < 1.26 - 1.25$, which immediately tells us that $\epsilon < 0.01$. Hence,
    $$
    f(n) = n^{1.25} \in \O \left( n^{log_3 4} \right)
    $$
    Therefore, by applying Master's theorem, we get that 
    $$
    T(n) \in \Theta \left( f(n) \right) \implies T(n) \in \Theta \left( n^{1.25} \right)
    $$
    So, finally
    $$
    T(n) \in \Theta \left( n^{1.25} \right)
    $$
\end{solution}

\begin{problem}
    Solve the following recurrence: $T(n)=T(n)+\log n$
\end{problem}
\begin{solution}
    Consider 
    $$
    T(n) = T(n) + \log n
    $$

    Expanding the given recurrence, one arrives at 

    \begin{align*}
        T(n) &= T(n) + \log n \\
        T(n) &= (T(n) + \log n) + \log n = T(n) + 2 \log n \\
        T(n) &= (T(n) + \log n) + 2 \log n = T(n) + 3 \log n \\
        &\cdots \\
        T(n) &= T(n) + k \log n \tag{$\forall k \in \mathbb N$}
    \end{align*}

    Since $k \in \mathbb N$, this expression is unbounded (divergent), and one can
    informally write
    $$
    T(n) = \infty
    $$
\end{solution}

\section{Happy Matrix Problem}

\begin{solution}
    The discussion will concern itself with an $R \times C$ matrix.
    The solution uses a divide-and-conquer approach to find the largest element (the happy element) on row $\frac R 2$, say at some column $j$.
    Now, we will consider the subproblems of the matrices of $\frac{R}{2} \times j$ and $\frac{R}{2} \times (C - j)$

    \[
    \left [
    \begin{array}{c c c|c c}
        a_{11} & \ldots &  a_{1j} & \ldots & a_{1C} \\
        \vdots & \ddots & \vdots & \ddots & \vdots \\
        a_{\frac{R}{2}1} & \ldots & a_{\frac{R}{2}j} & \ldots & a_{\frac{R}{2}C} \\
        \vdots & \ddots & \vdots & \ddots & \vdots \\
        a_{R1} & \ldots & a_{Rj} & \ldots & a_{RC}
    \end{array}
    \right ]
    \]

    The complexity to search for the correct $jth$ element in row $\frac{R}{2}$ will be $O(C)$ where $C$ is the size of the
    matrix we are interested in. We will also need to solve two matrices of the size $\frac{R}{2} \times j$ and $\frac{R}{2} \times (C - j)$.

    The recurrence will be
    $$
    T(R, C) = T\left(\frac{R}{2}, j\right) + T\left(\frac{R}{2}, C - j + 1\right) + O(C)
    $$

    This can also be written with a little bit of algebra as

    $$
    T(R, C) = T\left(\frac{R}{2}, \frac{C}{2} + \delta\right) + T\left(\frac{R}{2}, \frac{C}{2} - \delta + 1\right) + O(C)
    $$
    where $\delta = j - \frac{C}{2}$


    When the number of rows is $1$, then we will simply linear search on the columns to arrive at the recurrence bound
    $$
    T(1, C) = O(C)
    $$


    We will now prove a lemma which provides tight bounds on the above expression.
    \begin{lemma}
        For the given recurrence relation, 

        \begin{gather*}
            \forall r \in \mathbb N, \forall c \in \mathbb N, \text{ there exists an $f(c) \in O(c)$ such that} \\
            T \left(r, \frac{c}{2} + \delta\right) + T \left(r, \frac{c}{2} - \delta + 1 \right) \leq 2T \left(r, \frac{c}{2} \right) + f(c)
        \end{gather*}
    \end{lemma}
    \begin{proof}

        Consider strong induction over $r$.


        \begin{align*}
            \text{When $r = 1$, the left hand side is } \\
            T \left(1, \frac{c}{2} + \delta\right) + T \left(1, \frac{c}{2} - \delta + 1 \right) &=  \\
            \frac{c}{2} + \delta + \frac{c}{2} - \delta + 1 & = 2 \cdot \frac{c}{2} + 1 \\
            &= c + 1 \tag{(since T(1, c) = c)} \\
        \end{align*}
         
        and the right hand side is
        
        \begin{gather*}
            2T \left(1, \frac{c}{2} \right) + f(c) = 2 \cdot \frac c 2 + f(c) = \\
            c + f(c) \\
            \text{pick $\forall c, f(c) > 1$ so that we are able to get the inequality } \\
            c + 1 < c + f(c) \\
            \text{and hence} \\
            T \left(1, \frac{c}{2} + \delta\right) + T \left(1, \frac{c}{2} - \delta \right) \leq 2T \left(1, \frac{c}{2} \right) + O(c)
        \end{gather*}
        Therefore, he lemma holds for $r = 1$


        Now, assume by strong induction that it holds for all $r < R$. We need to show that it holds for $r = R$.

        Considering the expression $T \left(R, \frac{c}{2} + \delta \right)$ and breaking it down using the recurrence, 
        $$
        T \left(r, \frac{c}{2} + \delta \right) = T\left(\frac{R}{2}, \frac{\frac{c}{2} + \delta}{2} + \epsilon \right) + T\left(\frac{R}{2}, \frac{\frac{c}{2} + \delta}{2} - \epsilon + 1 \right) + O(c)
        $$
        Using the strong induction expression on the RHS terms yields
        \[
        T\left(\frac{R}{2}, \frac{\frac{c}{2} + \delta}{2} + \epsilon \right) + 
        T\left(\frac{R}{2}, \frac{\frac{c}{2} + \delta}{2} - \epsilon + 1 \right) \leq 2T\left(\frac{R}{2}, \frac{c}{2} + \delta \right) + f(c) \tag*{$f(c) \in O(c)$}
        \]

        Now, replacing the definition of
        {\small
        $ T\left(\frac{R}{2}, \frac{\frac{c}{2} + \delta}{2} + \epsilon \right) + T\left(\frac{R}{2}, \frac{\frac{c}{2} + \delta}{2} - \epsilon + 1 \right) + O(c) $
        }
        in
        {\small $T \left(r, \frac{c}{2} + \delta \right)$ }
        gives

        $$
        T \left(R, \frac{c}{2} + \delta \right)  \leq 2T\left(\frac{R}{2}, \frac{c}{2} + \delta \right) + f(c) + O(c)
        $$
        Since $f(c) \in O(c)$, we can safely absorb it in $O(c)$ to get
        $$
        T \left(R, \frac{c}{2} + \delta \right)  \leq 2T\left(\frac{R}{2}, \frac{c}{2} + \delta \right) + O(c)
        $$
        Hence, our theorem is also proved for $r = R$.
    \end{proof}

    Now, using the Lemma on our original expression
    $$
    T(R, C) = T\left(\frac{R}{2}, \frac{C}{2} + \delta\right) + T\left(\frac{R}{2}, \frac{C}{2} - \delta + 1\right) + O(C)
    $$
    
    Gives us

    \begin{gather*}
   	T(R, C) \leq 2 T \left(\frac{R}{2}, \frac{C}{2} \right) + f(C) +  O(C) \tag{$f(C) \in O(C)$}
    \end{gather*}
    
    Since $f(C) \in O(C)$, we can absorb it into $O(C)$ to arrive at 
    
    $$
	T(R, C) \leq 2 T \left(\frac{R}{2}, \frac{C}{2} \right) + O(C)
    $$
    
    Since $R = C$ in the beginning, and since we break down both $R$ and $C$ uniformly, we can replace $R = C = N$ giving
    $$
    T(N) \leq  2 T \left( \frac{N}{2} \right) + O(N) \\
    $$


    Which by an immediate application of Master's theorem yields $$T(N) \leq N \log N$$

    This means that the 
    algorithm has a complexity of $N \log N$ where $N$ is the size of the square matrix.
\end{solution}

\section{$n$ medians}
\subsection{Naive Solution}
The naive solution is to simply merge sort of the $k$ arrays into one large chunk, and then
take the middlemost element. This will have a time complexity of $O(n \cdot k)$ to merge all the $k$ arrays together.

One slight improvement is to notice that we don't need to take all $nk$ elements - simply taking
$n\frac{k}{2}$ will give us the median. Hence, the ``better'' time complexity is $O(\frac{n \cdot k}{2})$
\subsection{Better Solution}
\begin{solution}
I shall explain the general solution, which can then be specialised into the specific case that we are interested in ($2$ arrays).

We have $k$ arrays of size $n$ each. One property that we can rely on is that the median of all of the arrays will lie in between
the median of the individual arrays. Hence, we will get a smaller search space of that between the smallest median and the largest median.

To find the smallest and largest medians, we will first need to sort all the medians - this will take $O(k \log k)$ time.

Next, we pick an element in the range of the medians, and then check how many elements in the other arrays satisfy this. To do this,
we will binary search through all the arrays ($k$ of them) checking for the ``tipping point'' in all the arrays using binary search. This 
will take $k \log n$ time ($\log n$ to search for the tipping point in one array, $k$ to search in every array). If the element we picked turns
out to be the median, then we're good. If not, we need to repeat the search - and this will be done $k \log n$ times - we will binary search through the
range of positions in our $k$ arrays for good medians (the binary search takes $\log n$ time.

So, the split up is

\begin{align*}
\text{First: sort arrays according to medians} &: k \log k \\
\text{Loop: pick up all good medians} &: k \log n \\
\text{for one potential median, check if if is true median} &: k \log n 
\end{align*}

Hence, total time complexity is $O(k \log k + (k \log n) \cdot (k \log n)) = O(k \log k + (k \log n)^{2})$
 
\end{solution}
\end{document}
