\documentclass[12pt]{article}
\usepackage{amsmath, amsthm, amssymb}
\usepackage{mathtools}

\newtheorem*{problem}{Problem}
\theoremstyle{definition}
\newtheorem*{solution}{Solution}
\newtheorem{lemma}{Lemma}

\setcounter{secnumdepth}{5} 

\DeclarePairedDelimiter\floor{\lfloor}{\rfloor}


\begin{document}
\title{Assignment 3}
\author{Siddharth Bhat}
\date{\today}
\maketitle

\section{Prove or disprove the given statements}
\begin{problem}
Prove or disprove: $f(n) = O(g(n)) \implies 2^{f(n)} = O(2^{g(n)})$
\end{problem}
\begin{solution}
	Assume the given statement is true. 
	Consider $f(n) = 2n$, $g(n) = n$. Clearly, $f(n) \in O(g(n))$.
	Now,
	\begin{gather*}
		2^{f(n)} = 2^{2n}= (2^2)^n = 4^n \\
		O(2^{g(n)}) = O(2^n) \\
		\text{Since from our hypothesis} \\
		2^{f(n)} \in O(2^{g(n)}) \\
		4^n \in O(2^n) \\
		\text{ This implies that } \ \ \exists N \in \mathbb N,\ \exists c \in \mathbb R \text{ where $c$ is a constant, such that }
		\forall i \geq N, \ 4^n = c \cdot 2^n \\
		\left ( \frac{4}{2} \right ) ^n = c \implies \\
		c = 2^n \\
		\text{Hence, $c$ is not a constant, violating our assumption.}
	\end{gather*}
	
	Therefore, the theorem is incorrect.
\end{solution}

\begin{problem}
Prove or disprove: $f(n) + g(n) = \Theta(min \{ f(n), g(n) \})$
\end{problem}
\begin{solution}
	Given statement is false.
	
	let $f(n) = n$, $g(n) = 2^n$. Consider the problem statement to be
	true. This gives us
	
	\begin{gather*}
		f(n) + g(n) = n + 2^n \\
		\text{from the theorem, } \\
		f(n) + g(n) \in \Theta(min \{ f(n), g(n) \}) = \Theta(min \{n, 2^n \}) = \Theta(n) \\
		\text{which implies that} \\
		n + 2^n \in O(n) \\
		\text{which is false, since $2^n$ grows rapidly faster than $n$, which can
		be checked with a little algebra.}
	\end{gather*}
	Hence, by contradiction, the statement $f(n) + g(n) = \Theta(min \{ f(n), g(n) \})$ is false.
\end{solution}


\begin{problem}
Prove or disprove: $f(n) \neq O(g(n)) \implies g(n) = O(f(n))$
\end{problem}
\begin{solution}
	Given statement is false.
	
	Assume the statement is true. let $f(n) = \sin(n)$, $g(n) = \cos(n)$.
	
	$f(n) \neq O(g(n))$ since they are periodic and intersect each other
	infinitely many times over $\mathbb R$. Hence, they do not dominate
	one another. Similary, $g(n) \neq O(f(n))$ by the exact same reasoning.
\end{solution}


\begin{problem}
Prove or disprove: $min \{ f(n), g(n) \} \in \Theta(f(n) + g(n))$
\end{problem}
\begin{solution}
	Given statement is false.
	
	Assume the statement is true. Let $f(n) = n$, and $g(n) = n^2$. Consider the expression
	
	\begin{gather*}
	\forall n \in \mathbb N \\
	min \{f(n), g(n) \} = min \{n, n^{2} \} = n \\
	\text{and also} \\
	\Theta(f(n) + g(n)) = \Theta(n + n^{2}) = \Theta(n^{2})
	\end{gather*}
But we know that
$$
n \notin \Theta(n^{2})
$$

This violates our assumption that 
$$
min \{ f(n), g(n) \} \in (\Theta(f(n) + g(n))
$$
Hence, the statement is false
\end{solution}

\section{Solve the following recurrence relations}
\begin{problem}
Solve the following recurrence relation: $T(n) = 3T(\frac n 4) + T(\frac n 8) + n \log n$
\end{problem}
\begin{solution}
	TODO
\end{solution}

\begin{problem}
Solve the following recurrence relation: $T(n) = T(n - 10) + log(n)$
\end{problem}
\begin{solution}
	I make the assumption that
	\begin{gather*}
		\forall i, 0 \leq i \leq 9, T(i) = 1
	\end{gather*}
	
	Let $n = 10k + r, k, r \in \mathbb N, \ 0 \leq r < 10$ (by division algorithm). Now, reducing the recurrence relation,
	\begin{gather*}
		T(n) = T(n - 10) + \log n \\ß
		T(n) = (T(n - 20) + \log n) + \log n = T(n - 2 \cdot 10) + 2 \log n \\
		\text{generalizing to $p \in \mathbb N$ gives us} \\
		T(n) = T(n - 10p) + p \log n  \\
		T(10k + r) = T(10k + r - 10p) + p \log n \\
		\text{let $p = k$ to solve the recurrence} \\
		T(10k + r) = T(10k + r -10k) + k \log n \\
		T(10k + r) = T(r) + k \log n \\
		T(10k + r) = 1 + k \log n \\
		\text{but $k = \frac{(n - r)}{10} = \floor*{\frac{n}{10}}$. Hence} \\
		T(n) = 1 + \floor*{\frac{n}{10}} \log n
	\end{gather*}
\end{solution}


\begin{problem}
Solve the following recurrence: $T(n) = 4T(\frac n 3) + n^{\frac 5 4}$
\end{problem}
\begin{solution}
	Applying master's theorem on the given problem, with respect to the standard form
	$$
	T(n) = aT \left( \frac{n}{b} \right) + f(n) \\
	$$
	where $a = 4$, $b = 3$, $f(n) = n^{\frac 5 4} = n^{1.25}$.
	To apply Masters's Theorem, we need to consider the comparison value
	$$n^{\log_b a} = n^{log_3 4} = n^{1.26...}$$
	Since
	$$
	f(n) = n^{1.25} \in O(n^{\log_b a - \epsilon} = n^{1.26 - \epsilon})
	$$
	We can show the existence of $\epsilon < 1.26 - 1.25$, which immediately tells us that $\epsilon < 0.01$. Hence,
	$$
	f(n) = n^{1.25} \in \O \left( n^{log_3 4} \right)
	$$
	Therefore, by applying Master's theorem, we get that 
	$$
	T(n) \in \Theta \left( f(n) \right) \implies T(n) \in \Theta \left( n^{1.25} \right)
	$$
	So, finally
	$$
	T(n) \in \Theta \left( n^{1.25} \right)
	$$
\end{solution}

\begin{problem}
Solve the following recurrence: $T(n)=T(n)+\log n$
\end{problem}
\begin{solution}
	Consider 
	$$
	T(n) = T(n) + \log n
	$$
	
	Expanding the given recurrence, one arrives at 
	
	\begin{gather*}
		T(n) = T(n) + \log n = \\
		(T(n) + \log n) + \log n = T(n) + 2 \log n = \\
		(T(n) + \log n) + 2 \log n = T(n) + 3 \log n = \\
		\cdots \\
		T(n) = T(n) + k \log n, \forall k \in \mathbb N
	\end{gather*}
	
	Since $k \in \mathbb N$, this expression is unbounded (divergent), and one can
	informally write
	$$
	T(n) = \infty
	$$
\end{solution}

\section{Happy Matrix Problem}

\begin{solution}
The discussion will concern itself with an $R \times C$ matrix.
The solution uses a divide-and-conquer approach to find the largest element (the happy element) on row $\frac R 2$, say at some column $j$.
Now, we will consider the subproblems of the matrices of $\frac{R}{2} \times j$ and $\frac{R}{2} \times (C - j)$

\[
\left [
\begin{array}{c c c|c c}
a_{11} & \ldots &  a_{1j} & \ldots & a_{1C} \\
\vdots & \ddots & \vdots & \ddots & \vdots \\
a_{\frac{R}{2}1} & \ldots & a_{\frac{R}{2}j} & \ldots & a_{\frac{R}{2}C} \\
\vdots & \ddots & \vdots & \ddots & \vdots \\
a_{R1} & \ldots & a_{Rj} & \ldots & a_{RC}
\end{array}
\right ]
\]

The complexity to search for the correct $jth$ element in row $\frac{R}{2}$ will be $O(C)$ where $C$ is the size of the
matrix we are interested in. We will also need to solve two matrices of the size $\frac{R}{2} \times j$ and $\frac{R}{2} \times (C - j)$.

The recurrence will be
$$
T(R, C) = T\left(\frac{R}{2}, j\right) + T\left(\frac{R}{2}, C - j + 1\right) + O(C)
$$

This can also be written with a little bit of algebra as

$$
T(R, C) = T\left(\frac{R}{2}, \frac{C}{2} + \delta\right) + T\left(\frac{R}{2}, \frac{C}{2} - \delta + 1\right) + O(C)
$$
where $\delta = j - \frac{C}{2}$


When the number of rows is $1$, then we will simply linear search on the columns to arrive at the recurrence bound
$$
T(1, C) = O(C)
$$


We will now prove a lemma which provides tight bounds on the above expression.
\begin{lemma}
For the given recurrence relation, 

\begin{gather*}
\forall r \in \mathbb N, \forall c \in \mathbb N, \text{ there exists an $f(c) \in O(c)$ such that} \\
T \left(r, \frac{c}{2} + \delta\right) + T \left(r, \frac{c}{2} - \delta + 1 \right) \leq 2T \left(r, \frac{c}{2} \right) + f(c)
\end{gather*}
\end{lemma}
\begin{proof}

Consider strong induction over $r$.


\begin{gather*}
\text{When $r = 1$, the left hand side is } \\
T \left(1, \frac{c}{2} + \delta\right) + T \left(1, \frac{c}{2} - \delta + 1 \right) =  \\
\frac{c}{2} + \delta + \frac{c}{2} - \delta + 1= 2 \cdot \frac{c}{2} + 1= c + 1 \; \text{ (since T(1, c) = c)} \\
\text{and the right hand side is} \\
2T \left(1, \frac{c}{2} \right) + f(c) = 2 \cdot \frac c 2 + f(c) = c + f(c) \\
\text{pick $\forall c, f(c) > 1$ so that we are able to get the inequality } \\
c + 1 < c + f(c) \\
\text{and hence} \\
T \left(1, \frac{c}{2} + \delta\right) + T \left(1, \frac{c}{2} - \delta \right) \leq 2T \left(1, \frac{c}{2} \right) + O(c)
\end{gather*}
Therefore, he lemma holds for $r = 1$


Now, assume by strong induction that it holds for all $r < R$. We need to show that it holds for $r = R$.

Considering the expression $T \left(R, \frac{c}{2} + \delta \right)$ and breaking it down using the recurrence, 
$$
T \left(r, \frac{c}{2} + \delta \right) = T\left(\frac{R}{2}, \frac{\frac{c}{2} + \delta}{2} + \epsilon \right) + T\left(\frac{R}{2}, \frac{\frac{c}{2} + \delta}{2} - \epsilon + 1 \right) + O(c)
$$
Using the strong induction expression on the RHS terms yields
\[
T\left(\frac{R}{2}, \frac{\frac{c}{2} + \delta}{2} + \epsilon \right) + 
T\left(\frac{R}{2}, \frac{\frac{c}{2} + \delta}{2} - \epsilon + 1 \right) \leq 2T\left(\frac{R}{2}, \frac{c}{2} + \delta \right) + f(c) \tag*{$f(c) \in O(c)$}
\]

Now, replacing the definition of
{\tiny
$ T\left(\frac{R}{2}, \frac{\frac{c}{2} + \delta}{2} + \epsilon \right) + T\left(\frac{R}{2}, \frac{\frac{c}{2} + \delta}{2} - \epsilon + 1 \right) + O(c) $
}
in
{\tiny $T \left(r, \frac{c}{2} + \delta \right)$ }
gives

$$
T \left(R, \frac{c}{2} + \delta \right)  \leq 2T\left(\frac{R}{2}, \frac{c}{2} + \delta \right) + f(c) + O(c)
$$
Since $f(c) \in O(c)$, we can safely absorb it in $O(c)$ to get
$$
T \left(R, \frac{c}{2} + \delta \right)  \leq 2T\left(\frac{R}{2}, \frac{c}{2} + \delta \right) + O(c)
$$
Hence, our theorem is also proved for $r = R$.
\end{proof}

Now, using the Lemma on our original expression
$$
T(R, C) = T\left(\frac{R}{2}, \frac{C}{2} + \delta\right) + T\left(\frac{R}{2}, \frac{C}{2} - \delta + O(C) \right) + O(C)
$$

Gives us

$$
T(R, C) \leq 2 T \left(\frac{R}{2}, \frac{C}{2} \right) + O(C)
$$

Since $R = C$ in the beginning, and since we break down both $R$ and $C$ uniformly, we can replace $R = C = N$ giving
$$
T(N) \leq  2 T \left( \frac{N}{2} \right) + O(N) \\
$$


Which by an immediate application of Master's theorem yields $$T(N) \leq N \log N$$

This means that the 
algorithm has a complexity of $N \log N$ where $N$ is the size of the square matrix.
\end{solution}
\end{document}