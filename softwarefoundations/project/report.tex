\documentclass{article}
\usepackage{amsthm}
\usepackage{amsmath}
\usepackage{amssymb}
\usepackage{minted}
\newcommand{\N}{\mathbb{N}}
\usepackage[colorlinks]{hyperref}
\begin{document}
\author{Siddharth Bhat}
\title{Project Abstract: Formally verifying Dining Philosophers as Feedback Control}
\date{\today}
\theoremstyle{definition}
\newtheorem{definition}{Definition}

\theoremstyle{definition}
\newtheorem{lemma}[definition]{Lemma}

\maketitle

\section{Introduction}

I formally verify the first three sections of the paper 'Generalized Dining
Philosophers as Feedback Control'. Here, I describe the setting, definitions,
theorem statement, and a sketch of the proof. I then move to the experience
report, where I describe the interactions that occurred when attempting to
formally verify this in Coq. Finally, I end with a discussion of the feedback
given, as well as some thoughts on Tabuada composition as a substrate for
system design.


\section{Theorem Statement In Coq}



\subsection{System}
We first begin by defining a system.

\begin{definition} A \textbf{System} is a 4-tuple of (1) a ground set $X$ of system \textbf{states},
    (2) a set $U$ of \textbf{actions}, (3) a subset $X_0 \subseteq X$ of \textbf{initial states},
    (4) a relation $\rightarrow \subseteq X \times U \times X$ of \textbf{transitions}.
    Formally:
$$
\text{System} \equiv (X: \textbf{Set}, U: \textbf{Set}, X_0 \subseteq X, \rightarrow \subseteq X \times U \times X).
$$
\end{definition}

To encode this definition within Coq, note that we need to encode sets and
subsets. A set $S$ ie enoded by a type \texttt{T\_S}, whose values \texttt{v\_s : T\_S}
corresponds to members $s \in S$. To encode subsets, we exploit the fact that
a subset $X_0 \subseteq X$ is equivalent to a membership function
$\texttt{member}(X_0): X \rightarrow \texttt{Bool}$:


$$
X_0 \subseteq X \iff \texttt{member}(X_0): X \rightarrow \texttt{Bool}; ~ 
\texttt{member}(X_0)(x) \equiv 
\begin{cases}
    \texttt{true} & x \in X_0 \\
    \texttt{false} & x \not \in X_0
\end{cases}
$$

So, we encode a system in Coq as:
\begin{minted}{coq}
Record system (X: Set) (U: Set) := 
    mksystem { isx0: X -> Prop; trans: X -> U -> X -> Prop }.
\end{minted}

\subsection{Traces}

We now have the definition of a system. We need to define what a legal
trace of the given system is. Intuitively, a trace is a sequence of
states $xs : \N \rightarrow X$ and actions $us: \N \rightarrow U$,
such that (1) the first state is an initial state: $xs(0) \in X_0$; 
(2) successive states are related by the transition relation:
{$$\forall n \in \N, xs(n) \xrightarrow{us(n)} xs(n+1)$$}.

We encode this as an inductive proposition in coq called \texttt{ValidTrace s xs us n},
which witnesses the fact that $(xs, us)$ forms a valid trace for the system $s$
for $n$ steps:

\begin{minted}{coq}
(* ValidTrace s xs us n: trace suggested by xs, us is valid for n steps *)
Inductive ValidTrace {X U: Set}
    (s: system X U) (xs: nat -> X) (us: nat -> U): nat -> Prop :=
| Start: forall  (VALID: (isx0 X U s) (xs 0)) , ValidTrace s xs us 0
| Cons: forall (n: nat)
    (TILLN: ValidTrace s xs us n) (* Trace valid till n *)
    (ATN: trans X U s (xs n) (us n) (xs (S n))), (* Transition at n *)
    ValidTrace s xs us (S n).
\end{minted}

\subsection{Tabuada Interconnect}

Given two systems $S, T$, we define a new system, their \textbf{Tabuada Interconnect},
styled as $S \times_{\mathcal I} T$. This interconnect is governed by $\mathcal I$.
Formally:

\begin{itemize}
    \item[1]
        $S \equiv (X: \textbf{Set}, U_X: \textbf{Set}, X_0 \subseteq X, \underset{X}{\rightarrow} \subseteq X \times U_X \times X)$.
    \item[2] 
        $T \equiv (Y: \textbf{Set}, U_Y: \textbf{Set}, Y_0 \subseteq Y, \underset{Y}{\rightarrow} \subseteq Y \times U_Y \times Y)$.

    \item[3] \textbf{Interconnect:}
        $\mathcal I \subseteq (X \times Y) \times (U_X \times U_Y)$
    \item[4] \textbf{Composition:}
\begin{align*}
    &S \times_{\mathcal I} T \equiv 
     (Z \equiv X \times Y, U_Z \equiv U_X \times U_Y,  X_0 \times Y_0, \underset{Z, \mathcal{I}}{\rightarrow} \subseteq Z \times U_Z \times Z). \\
    &(x, y) \underset{Z, \mathcal I}{\xrightarrow{u_x,u_y}} (x', y') \iff 
      x \xrightarrow{u_x} x' \land 
      y \xrightarrow{u_y} y' \land (x, y, u_x, u_y) \in \mathcal I.
\end{align*}
\end{itemize}

In Coq, we define a new system as \mintinline{coq}{ tabuada sx sy connect : system}.

\begin{minted}{coq}
(* 2.2: system composition *)
(* tabuada connection new system *)
Definition tabuada {X Y UX UY: Set} 
  (sx: system X UX) (sy: system Y UY)
  (connect: X*Y->UX*UY->Prop): system (X*Y) (UX*UY) :=
  mksystem (X*Y) (UX*UY) (tabuada_start (isx0 X UX sx) (isx0 Y UY sy))
           (tabuada_trans connect (trans X UX sx) (trans Y UY sy)).
\end{minted}

Where \texttt{tabuada\_start} identifies the initial states of $\texttt{sx} \times_{\texttt{connect}} \texttt{sy}$,
and \texttt{tabuada\_trans} identified the transition:
\begin{minted}{coq}
(* initial state for tabuada composition *)
Definition tabuada_start {X Y: Type}
  (isx0: X -> Prop) (isy0: Y -> Prop) (x: X * Y): Prop :=
  isx0 (fst  x) /\ isy0 (snd x).

(* transition fn for tabuada composition *)                                                            
Definition tabuada_trans {X Y: Type} {UX UY: Type}
    (connect: X*Y->UX*UY->Prop)
    (transx: X -> UX -> X -> Prop)
    (transy: Y -> UY -> Y -> Prop)
    (s: X*Y) (u: UX*UY) (s': X*Y): Prop :=
  transx (fst s) (fst u) (fst s') /\
  transy (snd s) (snd u) (snd s') /\
  (connect s u).
\end{minted}


\begin{minted}{coq}
Lemma system38_starvation_free:
  forall (n: nat) (ss: nat -> the * cmd)
   (ts: nat -> cmd * maybe choice * the)
   (TRACE_SSSN: ValidTrace system38 ss ts (S(S(S n))))
   (BOTTOM_EVEN: 
      forall (i: nat) (IEVEN: even i = true), 
      snd (fst (ts i)) = nothing choice)
   (NOT_BOTTOM_ODD: 
      forall (i: nat) (IODD: odd i = true),
      snd(fst (ts i)) <> nothing choice)
   (HUNGRY: fst (ss n) = h),
  exists (m: nat),  m > n /\  fst (ss m) = e.
Proof.
    (* 30 lines of proof, 
       200 lines of supporting lemmas ommitted *)
Qed.

Lemma system_38_phil_not_hungry_then_next_philo_choice:
  forall (n: nat) (ss: nat -> the * cmd) (c: choice) 
   (ts: nat -> cmd * maybe choice * the)
   (TRACE_SSSN: ValidTrace system38 ss ts (S(S(S n))))
   (NOTHUNGRY: fst (ss (S n)) <> h)
   (CHOICE: snd (fst (ts (S n))) = just choice c)
   (NEVEN: even n = true)
   (BOTTOM_EVEN: 
     forall (i: nat) (IEVEN: even i = true), 
     snd (fst (ts i)) = nothing choice),
    fst (ss (S(S(S n)))) = trans32fn (fst (ss (S n))) c.
Proof.
    (* 20 lines of proof, 
       200 lines of supporting lemmas ommitted *)
Qed.
  
\end{minted}

\section{Proof Statistics}
\begin{itemize}
        \item  667 lines of coq code.
        \item All tables upto section 4 verified by computation.
        \item All theorems upto section 4 formally verified.
\end{itemize}

\section{Thoughts on tabuada composition}
\subsection{The Definition}
Traditionally, the composition $Z \equiv X \times_{\mathcal I} Y$ is defined as:

\begin{itemize}
    \item[1] \textbf{Interconnect:}
        $\mathcal I \subseteq (X \times Y) \times (U_X \times V_Y)$
    \item[2] \textbf{Composition:}
\begin{align*}
    &S \times_{\mathcal I} T \equiv 
     (Z \equiv X \times Y, U_Z \equiv U_X \times V_Y,  X_0 \times Y_0, \underset{Z, \mathcal{I}}{\rightarrow} \subseteq Z \times U_Z \times Z). \\
    &(x, y) \underset{Z, \mathcal I}{\xrightarrow{u_x,u_y}} (x', y') \iff 
      x \xrightarrow{u_x} x' \land 
      y \xrightarrow{u_y} y' \land (x, y, u_x, v_y) \in \mathcal I.
\end{align*}
\end{itemize}

Note that the definition of $\mathcal I$ does not discriminate between the
states $X \times Y$ and the actions $U_X \times U_Y$. However, if we consider
legal traces of this system:

$$
(x_0, y_0) \xrightarrow{(u_0, v_0)} (x_1, y_1) \xrightarrow{(u_1, v_1)} (x_2, y_2) \xrightarrow{(u_2, v_2)} \dots
$$

We note that since we \emph{start with the state} $(x_0, y_0)$, we are forced
to pick $(u_0, v_0) \in \{ (u, v) : (x_0, y_0, u, v) \in \mathcal I \}$. Similarly,
once we have determined a $(x_1, y_1)$, we are forced to pick $(u_1, v_1) \in \{ (u, v): (x_1, y_1, u, v) \in \mathcal I\}$. So,
there is an inherent  \emph{causality} in the way in which the tabuada
composition is \emph{applied}. Really, the type of $\mathcal I$ ought to be:

\begin{align*}
&\mathcal I: X \times Y \rightarrow 2^{U_X \times V_Y} \quad \text{(actions determined by state)}\\
&(x_n, y_n) \xrightarrow{(u_n, v_n)} (x_{n+1}, y_{n+1}) :  (u_n, v_n) \in \mathcal I(x_n, y_n)  \\
\end{align*}



\subsection{Control Using Tabuada Composition}

There appears 

\section{Feedback: Starting With Tabuada?}
The first bit of feedback was:
\begin{quote}
Seems that the problem statement has been
made unnecessarily complicated.  The starting
point could have been a coupled system of
equations that /assumes/ alternation.  That
way, the entire low level Tabuada composition
that invokes `bottom' as a possible value
could be skipped.  One has to carefully
choose the abstractions to coax Coq into
accepting them.
\end{quote}

As written in the abstract of the project proposal:

\begin{quote}
We propose to formally verify the proof development as sketched in the
paper on ArXiV: \href{https://arxiv.org/pdf/1805.02010.pdf}{Generalised Dining Philosophers as Feedback Control},
within the Coq proof assistant.
We will implement the proof, as laid out in sections
1, 2, and 3 (till \emph{The One Dining Philosopher Problem}) of the document
above.
\end{quote}

I followed the paper as it built the theory. My understanding was that
the key contribution of the paper was the modular reasoning offered by means
of composing systems using tabuada composition; a \emph{coupled} system of equations
is (by definition) \textbf{coupled} ---  ergo, monolithic. If we wished to begin from the coupled set of equations, I feel the problem
statement ought to have been framed differently.

On the other hand, I could have setup a  \emph{new system} which was driven by
a coupled set of equations, that I showed was equivalent to the original system
as defined by tabuada composition. I did not do this, since the lemmas
that I have proved (for driving the proof) are essentially equivalent
to having this "new system". It is unclear to me what the upshot of defining
this would be.Roving the equivalence between the original system and the new
system is equivalent to writing lemmas that "drive" the equational reasoning.



\section{Feedback: Encode Relations As Functions?}
The second bit of feedback was:

\begin{quote}
If relations are awkward to use in Coq, why
is the transition relation simply not defined
as a function of the type \texttt{State -> set State}?
I was unable to discern a clear reason.
\end{quote}

It's unclear this is a win. Specifically, \texttt{set} 
is not a \emph{free theory} --- equality of sets is not
freely generated from its constructors. Compare:

\begin{minted}{coq}
 Inductive  list (A: Type): Type :=
  nil : list A | cons : A -> list A -> list A.

Lemma list_eq_equational: forall
    (X: Type)
    (x x': X) (xs xs': list X)
    (EQ: cons X x xs = cons X x' xs'),
    x = x' /\ xs = xs'.
Proof.
  intros. inversion EQ. (* equational *)
  auto.
Qed.
\end{minted}

On the other hand, this theorem is patently false:

\begin{minted}{coq}
Lemma set_eq_equational: forall
    (X: Type)
    (x x': X) (xs xs': set X)
    (EQ: add X x xs = union X x' xs'),
    x = x' /\ xs = xs'.
Proof. Abort.
\end{minted}

For example, we know that:

\begin{align*}
\texttt{add}~ 1~ \{ 1 \} = \texttt{add}~1 ~\emptyset; \qquad \{ 1 \} \neq \emptyset
\end{align*}

Hence, we have no way to equationally reason about sets. We have
to carry proofs, and this burden of proof-carrying feels
equivalent to writing lemmas that perform computation. However,
there maybe "hidden wins" to doing it this way that I don't see; only trying
it out can actually tell.

\end{document}
